% This LaTeX document written 10/2015 by Andrew Wood.
% http://github.com/mongolsamurai/
%
% Written for and tested with pdflatex on Debian Linux
% pdfTeX 3.1415926-2.5-1.40.14 (TeX Live 2013/Debian)
%
% NOTE:
% Everything before "\begin{document}" is document parameters and custom
%   macros. It is not intended for editing, unless you know what you're doing.
%

\documentclass[12pt]{article}
\usepackage[hmargin=0.5in,vmargin=0.25in]{geometry}
\usepackage{wasysym}
\usepackage{amssymb}
\usepackage{multicol}
\usepackage{graphicx}
\usepackage{wrapfig}
\usepackage{placeins}

\setlength\parindent{0pt}

% Internal macros
\newcommand{\e}{\fullmoon}
\newcommand{\f}{\newmoon}
\newcommand{\spacer}{\hspace{4pt}\hrulefill\hspace{4pt}}
\newcommand{\logo}{\includegraphics[scale=0.25]{ex3_logo.jpg}}

% Symbol decorator macros
\newcommand{\N}{$\square$\hspace{4pt}}
\newcommand{\C}{$\sun$\hspace{4pt}}
\newcommand{\F}{$\heartsuit$\hspace{4pt}}
\newcommand{\s}{$\bigstar$\hspace{4pt}}
\newcommand{\0}{\e\e\e\e\e}
\newcommand{\1}{\f\e\e\e\e}
\newcommand{\2}{\f\f\e\e\e}
\newcommand{\3}{\f\f\f\e\e}
\newcommand{\4}{\f\f\f\f\e}
\newcommand{\5}{\f\f\f\f\f}

% Pool drawing macros
\newcommand{\pone}{\N}
\newcommand{\ptwo}{\N\N}
\newcommand{\pthree}{\N\N\N}
\newcommand{\pfour}{\N\N\N\N}
\newcommand{\pfive}{\N\N\N\N\N}
\newcommand{\pten}{$\square\square\square\square\square\square\square\square\square\square$}
\newcommand{\pfield}[1]{\rule[-3pt]{30pt}{0.4pt} / {#1}}

% Section and entry macros
\newcommand{\property}[2]{{\footnotesize#1}\hspace{8pt}\hrulefill\hspace{8pt}{\scriptsize\em#2}\normalsize\par}
\newcommand{\score}[2]{{\footnotesize#1\normalsize}\spacer{#2}\par}
\newcommand{\pool}[2]{{\footnotesize#1\normalsize}\hfill{#2}\par}
\newcommand{\sectionheader}[1]{\hrulefill\hspace{14pt}{#1}\hspace{14pt}\hrulefill}
\newcommand{\columnheader}[1]{\hrulefill\hspace{8pt}{\small\em#1}\hspace{8pt}\hrulefill\par}
\newcommand{\sheetheader}[2]{\raisebox{32px}{\LARGE#1}\hfill\logo\par\raisebox{24px}[0px][0px]{\em#2}\par\vspace{-20pt}}
\newcommand{\blankline}{\vspace{12pt}}
\newcommand{\nextcolumn}{\vfill\columnbreak}

%
% Character sheet contents are defined from here down.
%
% Note that everything that follows a '%' on a line is a comment for your
%   benefit, and won't affect the compiled sheet in any way. Triple comments
%   (%%%) are for readability only, they mark the end of a section or column.
%   You can add your own comments freely, e.g. to take notes during a game.
%
% Also note: LaTeX commands start with '\'. Nearly all of the commands below are
%   macros defined in the header section above. An itemized summary is provided
%   in comments at the bottom of this document, for your convenience 
%
% As a general rule, you can rearrange or remove sections of the character
%   sheet if needed. Indented sections usually represent a major segment of the
%   sheet. You can rearrange these chunks, but don't break them up. Contents
%   within a column can be rearranged, remove, or added to usually without any
%   trouble, but be aware of where your forced column breaks are (\nextcolumn).
%
% A final note for editors who aren't familiar with working with source code, I
%   highly recommend using a syntax-aware text editor for working with this
%   file. On Windows, a free and easy-to-install option is Notepad++. Mac has
%   numerous options, but I don't know any well enough to recommend one.
%   You'll want an editor that does syntax hilighting for LaTeX documents, it
%   makes this much easier to read.
%   For the love of god, don't use a word processor to work with this file,
%   it's very likely it will inject a bunch of crap that makes your LaTeX
%   compiler choke.
%

\begin{document}
  %%%
  \sheetheader{\spacer}{Caste:\spacer}
  %%%
  \sectionheader{Attributes}
  \begin{multicols}{3}
    \score{Strength}{\1}
    \score{Dexterity}{\1}
    \score{Stamina}{\1}
    %%%
    \score{Charisma}{\1}
    \score{Manipulation}{\1}
    \score{Appearance}{\1}
    %%%
    \score{Perception}{\1}
    \score{Intelligence}{\1}
    \score{Wits}{\1}
  \end{multicols}
  %%%
  \sectionheader{Abilities}
  % \s - supernal ability
  \begin{multicols}{3}
    {\N}\score{Archery}{\0}
    {\N}\score{Athletics}{\0}
    {\N}\score{Awareness}{\0}
    {\N}\score{Brawl}{\0}
    {\N}\score{Bureaucracy}{\0}
    {\N}\score{Dodge}{\0}
    {\N}\score{Integrity}{\0}
    {\N}\score{Investigation}{\0}
    {\N}\score{Larceny}{\0}
    {\N}\score{Linguistics}{\0}
    {\N}\score{Lore}{\0}
    {\N}\score{Medicine}{\0}
    {\N}\score{Melee}{\0}
    {\N}\score{Occult}{\0}
    {\N}\score{Performance}{\0}
    {\N}\score{Presence}{\0}
    {\N}\score{Resistance}{\0}
    {\N}\score{Ride}{\0}
    {\N}\score{Sail}{\0}
    {\N}\score{Socialize}{\0}
    {\N}\score{Stealth}{\0}
    {\N}\score{Survival}{\0}
    {\N}\score{Thrown}{\0}
    {\N}\score{War}{\0}
  \end{multicols}
  %%%
%
% NOTE: the next section (crafts and MA styles) is disable by default. This is
%   for characters who don't have either skill. If you need these fields, on
%   the next uncommented line change "\iffalse" to "\iftrue" to enable them.
%
% If you do, and you have lots of charms or merits, you may have to a problem
%   text overflowing onto the next page. If so, you may want to rearrange your
%   sections, or else you might need to push a section onto page 2. You can do
%   this cleanly by forcing a page break with the '\pagebreak' command.
%
  \iftrue
  \begin{multicols}{2}
    \columnheader{Crafts}
    {\N}\score{}{\0}
    {\N}\score{}{\0}
    {\N}\score{}{\0}
    \nextcolumn
    %%%
    \columnheader{Martial Arts}
    {\N}\score{}{\0}
    {\N}\score{}{\0}
    {\N}\score{}{\0}
    \vfill
  \end{multicols}
  \fi
  %%%
  \begin{multicols}{3}
    \columnheader{Specialties}
    \property{}{}
    \property{}{}
    \property{}{}
    \property{}{}
    \property{}{}
    \nextcolumn
    %%%
    \columnheader{Merits}
    \score{}{\0}
    \score{}{\0}
    \score{}{\0}
    \score{}{\0}
    \score{}{\0}
    \nextcolumn
    %%%
    \columnheader{Possessions}
    \property{}{}
    \property{}{}
    \property{}{}
    \property{}{}
    \property{}{}
  \end{multicols}
  %%%
  \sectionheader{Charms}
  \begin{multicols}{2}
    \property{}{}
    \property{}{}
    \property{}{}
    \property{}{}
    \property{}{}
    \property{}{}
    \property{}{}
    \property{}{}
    \property{}{}
    \property{}{}
    \property{}{}
    \property{}{}
    \property{}{}
    \property{}{}
    \property{}{}
    \property{}{}
    \property{}{}
    \property{}{}
  \end{multicols}
  %%%
  \setlength{\columnsep}{24pt}
  \begin{multicols}{3}
    \columnheader{Essence}
    \pool{Essence}{\1}
    \pool{Personal Essence}{\pfield{\rule[-3pt]{24pt}{0.4pt}}}
    \pool{Peripheral Essence}{\pfield{\rule[-3pt]{24pt}{0.4pt}}}
    \pool{Committed Essence}{\pfield{\rule[-3pt]{24pt}{0.4pt}}}
    \blankline
    \pool{Willpower}{\5\0}
    \pool{Temp Will}{\pten}
    \nextcolumn
    \columnheader{Combat Values}
    \property{Join Battle}{+ 3 successes}
    \property{Parry (rapier)}{}
    \property{Evasion}{}
    \property{Soak}{}
    \blankline
    \property{Resolve}{}
    \property{Guile}{}
    \nextcolumn
    \columnheader{Health Pool}
    \pool{-0}{\pfour}
    \pool{-1}{\pfour}
    \pool{-2}{\pfour}
    \pool{-2}{\pfour}
    \pool{-4}{\pfour}
    \pool{Incap}{\pone}
  \end{multicols}
%
% This last section is currently unfinished, it will be the character sheet back page.
%
  \iffalse
  %%%
  \pagebreak
  \sectionheader{Intimacies}
  \begin{multicols}{2}
    \columnheader{Ties}
    \nextcolumn
    %%%
    \columnheader{Principles}
  \end{multicols}
  \fi
  %%%
\end{document}
%
% Macro Guide:
%
% - \sheetheader{Title}{Subtitle}: places a two-line title section, with the Ex3
%     logo on the right. I use the subtitle for stating the character's exalt
%     type and caste, but you can use it for whatever you like.
%
% - \sectionheader{Section Title}: places a broken horizontal rule with a title
%     in the center. It will expand to the width of the frame it's in, which
%     usually means the width of the page. It will shrink to fit in a column if
%     you put it into one.
%
% - \columnheader{Column Title}: like \sectionheader, except the title font is
%     smaller, and italicised.
%
% - \begin{multicols}{n}: starts a page section that's divided into n columns.
%     This command is part of the multicols package, I can't take credit for it.
%
% - \end{multicols}: ends the most recent open column-divided page section. This
%     command is part of the multicols package, I can't take credit for it.
%
% - \nextcolumn: forces text to flow to the next column over after this line,
%     instead of trying to divide text as evenly as possible between columns.
%
% - \score{Name}{Rating}: makes a generic Exalted-style score in something.
%     This will expand to width, like \sectionheader. You can put text in the
%     Rating field, but it probably won't look good. See \property.
%
% - \property{Name}{Value}: like \score, except it assumes the value is text,
%     instead of a bubble sequence. The font size is smaller, and italicized.
%
% - \C, \F, \s, \N: inserts a symbol indicating caste, favored, supernal, or
%     normal (i.e. non-favored) skills.
%
% - \0, \1...\5: creates a sequence of 5 bubbles with the correspending number
%     filled in black. If you need a 10-bubble sequence (e.g. for willpower),
%     these can be concatenated together ({\5\2 will create a score of 7}.
%
